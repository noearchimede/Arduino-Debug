Questa libreria serve a semplificare il debug di qualsiasi programma per le piattaforme Arduino.

La libreria offre\+:


\begin{DoxyItemize}
\item un sistema di log su più livelli (errori, avvisi, informazioni)
\item un sistema di debug che consente di\+:
\begin{DoxyItemize}
\item inserire breakpoints che possono essere \char`\"{}sbloccati\char`\"{} inviando un carattere qualsiasi tramite un monitor seriale;
\item inserire breakpoints con una durata predeterminata, che se non sbloccati tramite il monitor seriale si sbloccano automaticamente allo scadere del tempo
\item bloccare irrimediabilmente il programma in caso di errore fatale
\item impostare tramite il monitor seriale il valore di variabili di qualsiasi tipo aritmetico (bool, int e uint di qualsiasi dimensione, float/double)
\end{DoxyItemize}
\end{DoxyItemize}

Per una descrizione dettagliate dell\textquotesingle{}interfaccia vedi la pagina \hyperlink{class_debug}{Debug}. 



{\bfseries Utilitzzo delle funzioni di logging}

Le funzioni {\ttfamily info}, {\ttfamily warn} e {\ttfamily err} stampano dei messaggi sul monitor seriale e fanno accendere un led per un tempo che dipende dal tipo di messaggio. L\textquotesingle{}unica differenza tra di esse è il livello di importanza\+: i messaggi info possono essere disabilitati lasciando warn ed err, e warn lasciando err. Inoltre i messaggi err sono visivamente in rilievo nell\textquotesingle{}output.

Ogni messaggi ha un nome o un codice. Il nome è una stringa di testo, il codice è un numero che rappresenta univocamente il messaggio. Conviene usare il nome se si ha abbastanza memoria a disposizione e altrimenti il codie, che evidentemente rende più impegnativo leggere il file di log. Inoltre a ogni messaggio è possibile associare un valore di qualsiasi tipo aritmetico (bool, intero con o senza segno o decimale), che sarà stampato accanto al nome.

Scelta delle funzioni\+:
\begin{DoxyItemize}
\item {\ttfamily info(...)}\+: informazione sul corretto svolgimento del programma, es.\+: svolto un certo calcolo, nuovo sensore connesso, ...
\item {\ttfamily warn(...)}\+: avviso su un potenziale problema che non dovrebbe accadere ma non compromette irreversibilmente lo svolgimento del programma, es.\+: tentativo di connessione a un sensore già connesso, temperatura\+: -\/400°C
\item {\ttfamily err(...)}\+: errore, cioè avvenimento indesiderato che compromette, o potrebbe verosimilmente compromettere il programma, es.\+: my\+Float è nan, divisione per 0 
\end{DoxyItemize}