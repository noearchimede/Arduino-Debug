\hypertarget{_debug__funzioni__vuote_8cpp}{}\section{Riferimenti per il file Debug/\+Debug\+\_\+funzioni\+\_\+vuote.cpp}
\label{_debug__funzioni__vuote_8cpp}\index{Debug/\+Debug\+\_\+funzioni\+\_\+vuote.\+cpp@{Debug/\+Debug\+\_\+funzioni\+\_\+vuote.\+cpp}}


Funzioni vuote.  


{\ttfamily \#include \char`\"{}Debug.\+hpp\char`\"{}}\newline


\subsection{Descrizione dettagliata}
Funzioni vuote. 

Se si disabilita la classe (definendo {\ttfamily D\+E\+B\+U\+G\+\_\+\+D\+I\+S\+A\+B\+I\+L\+I\+TA}) al posto delle funzioni vere saranno chiamati questi \char`\"{}fantasmi\char`\"{}. Questo permette di disattivare il sistema di debug (guadagnando spazio in memoria e tempo nel\textquotesingle{}esecuzione del programma) senza dover togliere ogni chiamata alle sue funzioni dal programma.

\begin{DoxyDate}{Data}
17 luglio 2017, 19 agosto 2017 
\end{DoxyDate}
